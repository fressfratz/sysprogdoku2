\section{Introduction}
Part II of the systems programming lab consists of the implementation of a parser that is based on the work of part I that performes a lexicographical analysis and generates a sequence of tokens.\\
The main goals of this part are the understanding of where the parser takes place in a compiler and learning to implement and understand a recursive descent which implements a LL(1)-parser. This part will also focus on the verification of the source code with the target language’s grammar, an algorithm to annotate the tree, which abstracts the source code, as generated by the LL(1)-parser and verifies compatibility of the types of the nodes of the tree, as required by the language and an algorithm to generate code for the target platform, based on the annotated tree, which is an assembly dialect for an interpreter that’s provided to the students for testing purposes.
\\\\
Similar to part I the general conditions dictate that the implementation will be done in Linux with C++
and forbid to use any of the data-structures that the standard template library (STL) provides.

\subsubsection*{Structure of the documentation}
First we will shortly describe the functionality of a parser before we introduce our implementation and explain all components that we have implemented.